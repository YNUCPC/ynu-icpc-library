\documentclass[12pt,a4paper,dvipdfmx]{jsarticle}
% \documentclass[12pt,a4paper,landscape,twocolumn,dvipdfmx]{jsarticle}  % 用紙横向き、二段組

\usepackage[margin=15mm]{geometry}  % 余白調整
\usepackage{fancyhdr}
\usepackage{listings,jlisting}
\usepackage{color}
\usepackage{amsmath,amssymb}
\usepackage{lastpage}

% ヘッダー・フッターの設定
\pagestyle{fancy}
\fancyhf{}
\rhead{\thepage/\pageref{LastPage}}  % ヘッダー右
\lhead{Yokohama National University}  % ヘッダー左
% \rfoot{}  % フッター右

% コードブロックの設定
\lstset{
  language={C++},
  backgroundcolor={\color[RGB]{250,250,250}}, % 背景色
  basicstyle={\ttfamily\small}, % 標準の書式. フォントサイズなど
  identifierstyle={}, % キーワード以外の書式
  commentstyle={\itshape \color[RGB]{0,128,0}},% コメントの書式
  keywordstyle={\bfseries \color{blue}},% キーワードの書式
  otherkeywords={constexpr}, % キーワードの追加
  stringstyle={\ttfamily \color[RGB]{163,21,21}}, % 文字列の書式
  directivestyle={\color[RGB]{128,128,128}},
  showstringspaces=false, % 文字列中の空白の表示
  frame=single, % 枠の追加
  breaklines=true, % 自動改行
  columns=fixed,%
  basewidth=0.5em,
  xrightmargin=0zw, %
  xleftmargin=3zw,%
  numbers=left, % 行番号表示位置
  numberstyle={\scriptsize}, % 行番号の書式
  stepnumber=1, % 行番号の表示行数間隔
  numbersep=1zw, % 行番号と本文の間隔
  lineskip=-0.5ex, % 行間隔
  tabsize=4,
}

\title{YNU ICPC Library}
\author{}
\date{}


\begin{document}

% \maketitle

% 目次の出力
\tableofcontents
\clearpage

% 本文
\section{テンプレート}

\lstinputlisting[firstline=3]{src/template/template.hpp}

\section{グラフ}

\subsection{Lowest Common Ancestor}

木の最近共通祖先(Lowest Common Ancestor: LCA)をダブリングにより求める。前計算:時間・空間ともに $O(V \log V)$、クエリあたり:$O(\log V)$ である。
\begin{itemize}
    \item LCA(G, r) : 木 $G$ と 根 $r$ から、前計算する。
    \item int query(u, v) : LCA($u,v$) を求める.
    \item bool is\_on\_path(u, v, a) : 頂点 $a$ が 頂点 $u, v$ を結ぶパス上に存在するかどうか
\end{itemize}

\lstinputlisting[firstline=5]{src/graph/lca.hpp}

\subsection{scc}

有向グラフを強連結成分分解する。計算量は $O(V + E)$
\begin{itemize}
    \item scc\_graph graph(int V) : コンストラクタ. $V$ 頂点 $E$ 辺の有向グラフを作る.
    \item void graph.add\_edge(int from, int to) : 頂点 from から 頂点 to へ有向辺を足す.
    \item pair\verb|<|
    \item vector\verb|<|vector\verb|<|int\verb|>>|  graph.scc() : \\ 次の条件を満たす「頂点のリスト」のリストを返す.
    \begin{itemize}
        \item 全ての頂点がちょうど 1 つずつ、どれかのリストに含まれる.
        \item 内側のリストと強連結成分が一対一に対応する. リスト内の順序は未定義.
        \item リストはトポロジカルソートされている.
    \end{itemize}
\end{itemize}

\lstinputlisting[firstline=5]{src/graph/scc.hpp}
\subsection{2-SAT}

$n$ 変数 $x_0, x_1, \dots, x_{n-1}$ に関して、
\begin{align*}
    (x_i = f) \vee (x_j = g)
\end{align*}
というクローズを足し、これを全て満たす変数の割り当てがあるか、という問題を解く。
\begin{itemize}
    \item two\_sat(n) : n 変数の 2-SAT を作る。$O(n)$
    \item void add\_clause(i, f, j, g) : クローズ $(x_i = f) \vee (x_j = g)$ を足す。ならし $O(1)$
    \item bool satisfiable() : \\ (割り当てが存在する ? true : false). クローズの個数を $m$ として $O(n + m)$
    \item vector\verb|<|bool\verb|>| answer() :\\ 最後に呼んだ satisfiable のクローズを満たす割り当てを返す。
    satisfiable を呼ぶ前や、割り当てがない場合、中身が未定義の長さ $n$ の vector を返す。$O(n)$
\end{itemize}

\lstinputlisting[firstline=6]{src/graph/two_sat.hpp}


\section{フロー}

\subsection{dinic}

最大流問題を解くアルゴリズム。計算量は$O(VE^2)$だが実用上かなり高速なことが多い。

\begin{itemize}
    \item Dinic flow(V): 構造体の宣言。Vは頂点数。
    \item flow.add$\_$edge(u, v, c): u → v に容量cの辺を追加する
    \item flow.max$\_$flow(s, t): s → tの最大流を返す
\end{itemize}

\lstinputlisting[firstline=5]{src/flow/dinic.hpp}
\subsection{最小費用流}

最小費用流問題を解くアルゴリズム。計算量は$O(FE\log V)$

\begin{itemize}
    \item MinCostFlow flow(V): 構造体の宣言。Vは頂点数。
    \item flow.add$\_$edge(u, v, c, d): u → v に容量c, コストdの辺を追加する
    \item flow.min$\_$cost$\_$flow(s, t, F): s → tに流量Fを流すときの最小コストを返す。流せない場合は-1を返す。
\end{itemize}

\lstinputlisting[firstline=5]{src/flow/mincostflow.hpp}
\section{数学}

\subsection{modint}

自動的にmodをとる構造体。modが問題で固定であるとき使用できる。

using mint = modint\verb|<|1000000007\verb|>|; 等のように定義して使用するのが推奨。

\lstinputlisting[firstline=4]{src/math/modint.hpp}
\subsection{hoge}

hogeを出力する

\lstinputlisting[firstline=5]{src/math/hoge.hpp}
\section{データ構造}

\subsection{セグメント木}

モノイドを満たすデータ$S$に対し使用できるデータ構造。

長さ$N$の$S$の配列に対し、要素の1点更新、区間クエリを$O(\log N)$で行える。
モノイド$S$同士の演算の計算量が$O(f(n))$とき、すべての計算量が$O(f(n))$倍になる。


\lstinputlisting[firstline=4]{src/datastructure/segtree.hpp}
\subsection{遅延評価セグメント木}

モノイド$S$と、$S$に対する作用素$f:S \rightarrow S$に対し利用できるデータ構造。

長さ$N$の$S$の配列に対し、
%
\begin{itemize}
    \item 区間$[l,r)$の要素に一括で$f$を作用($a_i \leftarrow f(a_i), l \le i < r$)
    \item 区間$[l,r)$の要素の総積の取得
\end{itemize}
%
を$O(\log N)$で行うことができる。

\lstinputlisting[firstline=4]{src/datastructure/lazysegtree.hpp}

\subsubsection{使用例}

Range Update \& Range Minimum Query

\begin{lstlisting}
constexpr int INF = INT32_MAX;
constexpr int ID = INT32_MAX;

int op(int a, int b) { return min(a, b); }
int e() { return INF; }
int mapping(int f, int a) { return (f == ID ? a : f); }
int composition(int f, int g) { return (f == ID ? g : f); }
int id() { return ID; }

int n;
LazySegmentTree<int, op, e, int, mapping, composition, id> seg(n);
\end{lstlisting}

Range Add \& Range Sum Query

Range Add \& Range Minimum Query

Range Update \& Range Sum Query


\end{document}