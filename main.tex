\documentclass[12pt,a4paper,dvipdfmx]{jsarticle}
% \documentclass[12pt,a4paper,landscape,twocolumn,dvipdfmx]{jsarticle}  % 用紙横向き、二段組

\usepackage[margin=15mm]{geometry}  % 余白調整
\usepackage{fancyhdr}
\usepackage{listings,jlisting}
\usepackage{color}
\usepackage{amsmath,amssymb}
\usepackage{lastpage}

% ヘッダー・フッターの設定
\pagestyle{fancy}
\fancyhf{}
\rhead{\thepage/\pageref{LastPage}}  % ヘッダー右
\lhead{Yokohama National University}  % ヘッダー左
% \rfoot{}  % フッター右

% コードブロックの設定
\lstset{
  language={C++},
  backgroundcolor={\color[RGB]{250,250,250}}, % 背景色
  basicstyle={\ttfamily\small}, % 標準の書式. フォントサイズなど
  identifierstyle={}, % キーワード以外の書式
  commentstyle={\itshape \color[RGB]{0,128,0}},% コメントの書式
  keywordstyle={\bfseries \color{blue}},% キーワードの書式
  otherkeywords={constexpr}, % キーワードの追加
  stringstyle={\ttfamily \color[RGB]{163,21,21}}, % 文字列の書式
  directivestyle={\color[RGB]{128,128,128}},
  showstringspaces=false, % 文字列中の空白の表示
  frame=single, % 枠の追加
  breaklines=true, % 自動改行
  columns=fixed,%
  basewidth=0.5em,
  xrightmargin=0zw, %
  xleftmargin=3zw,%
  numbers=left, % 行番号表示位置
  numberstyle={\scriptsize}, % 行番号の書式
  stepnumber=1, % 行番号の表示行数間隔
  numbersep=1zw, % 行番号と本文の間隔
  lineskip=-0.5ex, % 行間隔
  tabsize=4,
}

\title{YNU ICPC Library}
\author{}
\date{}


\begin{document}

% \maketitle

% 目次の出力
\tableofcontents
\clearpage

% 本文
\section{テンプレート}

\lstinputlisting[firstline=3]{src/template/template.hpp}

\section{グラフ}
\section{フロー}

\subsection{dinic}

最大流問題を解くアルゴリズム。計算量は$O(VE^2)$だが実用上かなり高速なことが多い。

\subsubsection*{宣言とメンバ関数}

\begin{enumerate}
    \item Dinic flow(V): 構造体の宣言。Vは頂点数。
    \item flow.add$\_$edge(u, v, c): u → v に容量cの辺を追加する
    \item flow.max$\_$flow(s, t): s → tの最大流を返す
\end{enumerate}

\lstinputlisting[firstline=5]{src/flow/dinic.hpp}
\subsection{最小費用流}

最小費用流問題を解くアルゴリズム。計算量は$O(FE\log V)$

\begin{itemize}
    \item MinCostFlow flow(V): 構造体の宣言。Vは頂点数。
    \item flow.add$\_$edge(u, v, c, d): u → v に容量c, コストdの辺を追加する
    \item flow.min$\_$cost$\_$flow(s, t, F): s → tに流量Fを流すときの最小コストを返す。流せない場合は-1を返す。
\end{itemize}

\lstinputlisting[firstline=5]{src/flow/mincostflow.hpp}
\section{数学}

\subsection{modint}

自動的にmodをとる構造体。modが問題で固定かつ素数であるとき使用できる。

using mint = modint\verb|<|1000000007\verb|>|; 等のように定義して使用するのが推奨。

\lstinputlisting[firstline=5]{src/math/modint.hpp}

\section{データ構造}

\subsection{Fenwick Tree}
一点更新・区間和取得のクエリを $O(\log N)$ で処理するデータ構造。
\begin{itemize}
    \item add(i, x) : 点更新 $a[i] \xleftarrow{} a[i] + x$
    \item sum(l, r) : 区間和取得 $a[l] + a[l + 1] + \cdots + a[r - 1]$
\end{itemize}
\lstinputlisting[firstline=5]{src/datastructure/fenwick_tree.hpp}

\subsection{セグメント木}

モノイドを満たすデータ$S$に対し使用できるデータ構造。

長さ$N$の$S$の配列に対し、要素の1点更新、区間クエリを$O(\log N)$で行える。
モノイド$S$同士の演算の計算量が$O(f(n))$とき、すべての計算量が$O(f(n))$倍になる。

\lstinputlisting[firstline=5]{src/datastructure/segtree.hpp}

\subsubsection*{使用例}

Range Minimum Query (RMQ)

\begin{lstlisting}
int op(int a, int b) { return min(a, b); }
int e() { return INT32_MAX; }

int n;
SegmentTree<int, op, e> seg(n);
\end{lstlisting}
\subsection{遅延評価セグメント木}

モノイド$S$と、$S$に対する作用素$f:S \rightarrow S$に対し利用できるデータ構造。

長さ$N$の$S$の配列に対し、
%
\begin{itemize}
    \item 区間$[l,r)$の要素に一括で$f$を作用($a_i \leftarrow f(a_i), l \le i < r$)
    \item 区間$[l,r)$の要素の総積の取得
\end{itemize}
%
を$O(\log N)$で行うことができる。

\lstinputlisting[firstline=4]{src/datastructure/lazysegtree.hpp}

\subsubsection{使用例}

Range Update \& Range Minimum Query

\begin{lstlisting}
constexpr int INF = INT32_MAX;
constexpr int ID = INT32_MAX;

int op(int a, int b) { return min(a, b); }
int e() { return INF; }
int mapping(int f, int a) { return (f == ID ? a : f); }
int composition(int f, int g) { return (f == ID ? g : f); }
int id() { return ID; }

int n;
LazySegmentTree<int, op, e, int, mapping, composition, id> seg(n);
\end{lstlisting}

Range Add \& Range Sum Query

Range Add \& Range Minimum Query

Range Update \& Range Sum Query

\subsection{Undo つき UnionFind}

経路圧縮を行わないことで undo 可能にした UnionFind。
\begin{itemize}
    \item RollbackUnionFind(n) : 大きさ $n$ の UnionFind を生成する。
    \item unite(x,y) : $x$ と $y$ のマージに成功したら true 失敗したら false を返す。 $O(\log{n})$
    \item find(x) : $x$ の根を返す。 $O(\log{n})$
    \item undo() : 直前の unite 操作を取り消す。 $O(1)$ 
    \item snapshot() : 現在の UnionFind の状態を保存する。$O(1)$
    \item state() : 現在までに unite() が呼ばれた回数を返す。 $O(1)$
    \item rollback(t) : 
    \begin{itemize}
        \item $t = -1$ のとき : snapshot() で保存した状態まで巻き戻す。
        \item $t \neq -1$ のとき : unite() が $t$ 回 呼び出された時の状態まで巻き戻す。
    \end{itemize}
\end{itemize}

\lstinputlisting[firstline=5]{src/datastructure/rollback_unionfind.hpp}


\end{document}