\subsection{セグメント木}

モノイドを満たすデータ$S$に対し使用できるデータ構造。

長さ$N$の$S$の配列に対し、要素の1点更新、区間クエリを$O(\log N)$で行える。
モノイド$S$同士の演算の計算量が$O(f(n))$とき、すべての計算量が$O(f(n))$倍になる。

\lstinputlisting[firstline=5]{src/datastructure/segtree.hpp}

\subsubsection*{使用例}

Range Minimum Query (RMQ)

\begin{lstlisting}
int op(int a, int b) { return min(a, b); }
int e() { return INT32_MAX; }

int n;
SegmentTree<int, op, e> seg(n);
\end{lstlisting}