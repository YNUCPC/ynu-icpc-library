\subsection{遅延評価セグメント木}

モノイド$S$と、$S$に対する作用素$f:S \rightarrow S$に対し利用できるデータ構造。

長さ$N$の$S$の配列に対し、
%
\begin{itemize}
    \item 区間$[l,r)$の要素に一括で$f$を作用($a_i \leftarrow f(a_i), l \le i < r$)
    \item 区間$[l,r)$の要素の総積の取得
\end{itemize}
%
を$O(\log N)$で行うことができる。

\lstinputlisting[firstline=4]{src/datastructure/lazysegtree.hpp}

\subsubsection{使用例}

Range Update \& Range Minimum Query

\begin{lstlisting}
constexpr int INF = INT32_MAX;
constexpr int ID = INT32_MAX;

int op(int a, int b) { return min(a, b); }
int e() { return INF; }
int mapping(int f, int a) { return (f == ID ? a : f); }
int composition(int f, int g) { return (f == ID ? g : f); }
int id() { return ID; }

int n;
LazySegmentTree<int, op, e, int, mapping, composition, id> seg(n);
\end{lstlisting}

Range Add \& Range Sum Query

Range Add \& Range Minimum Query

Range Update \& Range Sum Query
