\subsection{Aho-Corasick法}

複数文字列検索アルゴリズム。入力文字列$t$内の文字列パターン$\{s_1, s_2, \ldots, s_m \}$の要素と一致する箇所を全て検索する。Trie木を拡張したオートマトンを構築する。

\begin{itemize}
    \item insert(s) : パターンとして文字列$s$を追加する。計算量$O(|s|)$
    \item build() : 前処理としてオートマトンの状態遷移を構築する。計算量$O(k\cdot\sum{|s_i|})$。 $k$は文字の種類数。これ以下の関数はbuild()後に正常に動作する。
    \item next(idx) : 頂点idxから文字cで遷移したときの頂点番号を返す。計算量$O(1)$
    \item accept(idx) : 頂点idxにマッチするパターンが存在するか。
    \item match(idx) :  頂点idxにマッチするすべてのパターンのIDを返す。
    \item search(t) : 文字列$t$にマッチするパターンを全て求める。{\ttfamily res[i]}には$t[i]$を末尾としてマッチするパターンのIDリストが格納される。計算量$O(|t|+マッチ数)$
\end{itemize}

\lstinputlisting[firstline=5]{src/string/aho_corasick.hpp}
